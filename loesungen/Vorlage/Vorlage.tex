\documentclass[a4paper,12pt]{article}
\usepackage{fancyhdr}
\usepackage{fancyheadings}
\usepackage[ngerman,german]{babel}
\usepackage{german}
\usepackage[utf8]{inputenc}
%\usepackage[latin1]{inputenc}
\usepackage[active]{srcltx}
\usepackage{algorithm}
%\usepackage[noend]{algorithmic}
\usepackage{amsmath}
\usepackage{amssymb}
\usepackage{amsthm}
\usepackage{bbm}
\usepackage{enumerate}
\usepackage{graphicx}
\usepackage{ifthen}
\usepackage{listings}
\usepackage{struktex}
\usepackage{hyperref}

%%%%%%%%%%%%%%%%%%%%%%%%%%%%%%%%%%%%%%%%%%%%%%%%%%%%%%
%%%%%%%%%%%%%% EDIT THIS PART %%%%%%%%%%%%%%%%%%%%%%%%
%%%%%%%%%%%%%%%%%%%%%%%%%%%%%%%%%%%%%%%%%%%%%%%%%%%%%%
\newcommand{\Fach}{Algorithmen und Programmierung}
\newcommand{\Name}{Max Mustermann}
\newcommand{\Seminargruppe}{2}
\newcommand{\Matrikelnummer}{1337}
\newcommand{\Semester}{WS 12/13}
\newcommand{\Uebungsblatt}{0} %  <-- UPDATE ME
%%%%%%%%%%%%%%%%%%%%%%%%%%%%%%%%%%%%%%%%%%%%%%%%%%%%%%
%%%%%%%%%%%%%%%%%%%%%%%%%%%%%%%%%%%%%%%%%%%%%%%%%%%%%%

\setlength{\parindent}{0em}
\topmargin -1.0cm
\oddsidemargin 0cm
\evensidemargin 0cm
\setlength{\textheight}{9.2in}
\setlength{\textwidth}{6.0in}

%%%%%%%%%%%%%%%
%% Aufgaben-COMMAND
\newcommand{\Aufgabe}[1]{
  {
  \vspace*{0.5cm}
  \textsf{\textbf{Aufgabe #1}}
  \vspace*{0.2cm}
  
  }
}
%%%%%%%%%%%%%%
\hypersetup{
    pdftitle={\Fach{}: Übungsblatt \Uebungsblatt{}},
    pdfauthor={\Name},
    pdfborder={0 0 0}
}

\lstset{ %
language=java,
basicstyle=\footnotesize\tt,
showtabs=false,
tabsize=2,
captionpos=b,
breaklines=true,
extendedchars=true,
showstringspaces=false,
flexiblecolumns=true,
}

\title{Übungsblatt \Uebungsblatt{}}
\author{\Name{}}

\begin{document}
\thispagestyle{fancy}
\lhead{\sf \large \Fach{} \\ \small \Name{} - \Matrikelnummer{}}
\rhead{\sf \Semester{} \\  Seminargruppe \Seminargruppe{}}
\vspace*{0.2cm}
\begin{center}
\LARGE \sf \textbf{Übungsblatt \Uebungsblatt{}}
\end{center}
\vspace*{0.2cm}

%%%%%%%%%%%%%%%%%%%%%%%%%%%%%%%%%%%%%%%%%%%%%%%%%%%%%%
%% Insert your solutions here %%%%%%%%%%%%%%%%%%%%%%%%
%%%%%%%%%%%%%%%%%%%%%%%%%%%%%%%%%%%%%%%%%%%%%%%%%%%%%%

\Aufgabe{1}
\begin{enumerate}[a)]
    \item Formeln lassen sich einfach mit zwei \$-Zeichen den Text integrieren: $\frac{a + b}{c} = 1$ . Für eine Längere Formel, die in einer eigenen Zeile stehen soll können doppelte \$-Zeichen verwendet werden:
$$ \frac{a + b}{c} = 1 \Rightarrow a + b = c $$
    \item Nummerierte Formeln bietet die \emph{equation}-Umgebung:
        \begin{equation}
            (x+y)^n=\sum_{k=0}^n\binom{n}{k}x^ky^{n-k}
            \label{myequation}
        \end{equation}
        Später kann Formel \ref{myequation} referenziert werden. 
    \item Mehrzeilige Formeln können durch die \emph{align}-Umgebung realisiert werden:
        \begin{align*}
            ggT(15, 12) &= ggT(3, 12) \\
                        &= ggT(3, 9) \\
                        &= ggT(3, 6) \\
                        &= ggT(3, 3) \\
                        &= 3
        \end{align*}
    Eine sehr umfangreiche Hilfe zu Formeln in \LaTeX{} findet sich in \url{http://de.wikipedia.org/wiki/Hilfe:TeX}.
\end{enumerate}

\Aufgabe{2}
Quelltexte lassen sich ebenfalls einfach in \LaTeX{} einbetten:
\begin{lstlisting}
public class Forloop {
  public static void main(String[] args) {
    int factorial = 1;
    for (int count=1; count < 11; count++) {
       System.out.println(factorial *= count);
    }
  }
}
\end{lstlisting}

Auch direkt von einer Datei mit:
\begin{lstlisting}
\lstinputlisting{src/Forloop.java}
\end{lstlisting}

%%%%%%%%%%%%%%%%%%%%%%%%%%%%%%%%%%%%%%%%%%%%%%%%%%%%%%
%%%%%%%%%%%%%%%%%%%%%%%%%%%%%%%%%%%%%%%%%%%%%%%%%%%%%%
\end{document}


